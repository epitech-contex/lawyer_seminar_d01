%%%%%%%%%%%%%%%%%%%%%%%%%%%%%%%%%%%%%%%%%%%%%%%%%%%%%%%
%%%
%%% EPITECH LaTeX template
%%%                                    v2.1
%%%
%%%                Pierre ROBERT
%%%                May 2017
%%%
%%%%%%%%%%%%%%%%%%%%%%%%%%%%%%%%%%%%%%%%%%%%%%%%%%%%%%%

\documentclass
	[a4paper,
	 11pt     ,
	 babel-en ;
	]
{article}




%%%%%%%%%%%%%%%%%%%%%%%%%%%%%%%%%%%%%%%%%%%%%%%%%%%%%%%%%%%%%%%%%%%%%%%%%%%%%%
%
% packages
%
%%%%%%%%%%%%%%%%%%%%%%%%%%%%%%%%%%%%%%%%%%%%%%%%%%%%%%%%%%%%%%%%%%%%%%%%%%%%%%
\usepackage[top=4cm, bottom=3cm, left=2cm , right=2cm]{geometry}
\usepackage{courier}
\usepackage{listings}
\RequirePackage[unicode=true]{hyperref}
\RequirePackage{Epitech}
\RequirePackage{eso-pic}				% background image
\RequirePackage{titlesec} 				% section formatting
\RequirePackage{needspace}				% unbreakable (sub*)sections

\definecolor{mygreen}{rgb}{0,0.6,0}
\definecolor{mygray}{rgb}{0.5,0.5,0.5}
\definecolor{mymauve}{rgb}{0.58,0,0.82}

\lstset{ %
  basicstyle=\footnotesize\ttfamily,        % size of fonts used for the code
  breaklines=true,                 % automatic line breaking only at whitespace
  captionpos=b,                    % sets the caption-position to bottom
  commentstyle=\color{mygreen},    % comment style
  escapeinside={\%*}{*)},          % if you want to add LaTeX within your code
  keywordstyle=\color{blue},       % keyword style
  stringstyle=\color{mymauve},     % string literal style
}

\newcommand{\passthrough}[1]{#1}




%%%%%%%%%%%%%%%%%%%%%%%%%%%%%%%%%%%%%%%%%%%%%%%%%%%%%%%%%%%%%%%%%%%%%%%%%%%%%%
%
% general settings
%
%%%%%%%%%%%%%%%%%%%%%%%%%%%%%%%%%%%%%%%%%%%%%%%%%%%%%%%%%%%%%%%%%%%%%%%%%%%%%%
\setcounter{page}{0}
\thispagestyle{empty}

\hypersetup{breaklinks=true,
            pdfauthor={Epitech},
            pdfsubject={Initiation au développement informatique},
            pdftitle={Programme},
            pdfcreator={Raphael Fourdrilis},
            pdfkeywords={Epitech, Cap Gemini, Programme},
            colorlinks=true,
            citecolor=COLOR_light,
            urlcolor=COLOR_light,
            pdfborder={0 0 0}}



%%%%%%%%%%%%%%%%%%%%%%%%%%%%%%%%%%%%%%%%%%%%%%%%%%%%%%%%%%%%%%%%%%%%%%%%%%%%%%
%
% config
%
%%%%%%%%%%%%%%%%%%%%%%%%%%%%%%%%%%%%%%%%%%%%%%%%%%%%%%%%%%%%%%%%%%%%%%%%%%%%%%
	\newcommand{\moduleCode}{Cap Gemini}
	\newcommand{\projectTitle}{Programme}
	\newcommand{\projectSubtitle}{Initiation au développement informatique}
	\newcommand{\backgroundImage}{./png/EpitechCapBackground.png}

\providecommand{\tightlist}{%
  \setlength{\itemsep}{0pt}\setlength{\parskip}{0pt}}



%%%%%%%%%%%%%%%%%%%%%%%%%%%%%%%%%%%%%%%%%%%%%%%%%%%%%%%%%%%%%%%%%%%%%%%%%%%%%%
%
% background image
%
%%%%%%%%%%%%%%%%%%%%%%%%%%%%%%%%%%%%%%%%%%%%%%%%%%%%%%%%%%%%%%%%%%%%%%%%%%%%%%
\newcommand{\bgimg}
{
  	\AddToShipoutPicture
	{
		\put(\LenToUnit{0 cm},\LenToUnit{0 cm})
			{\includegraphics[decodearray={\FPprint{\redQtt} 1 \FPprint{\greenQtt} 1 \FPprint{\blueQtt} 1}]{\backgroundImage}}
	}
}



%%%%%%%%%%%%%%%%%%%%%%%%%%%%%%%%%%%%%%%%%%%%%%%%%%%%%%%%%%%%%%te%%%%%%%%%%%%%%
%
% section formatting
%
%%%%%%%%%%%%%%%%%%%%%%%%%%%%%%%%%%%%%%%%%%%%%%%%%%%%%%%%%%%%%%%%%%%%%%%%%%%%%%
\titlespacing{\section}{0px}{80px}{10px}
\titleformat{\section}
{\Needspace*{9\baselineskip}\LARGE\scshape\raggedright\bfseries}
{}{0em}{}
[\normalsize\vspace*{-14px}\_ \_ \_ \_ \_ \_ \hrulefill~\_ \_ \_ \_ \_\phantomsection]

\titlespacing{\subsection}{100px}{20px}{\baselineskip}
\titleformat{\subsection}
{\Needspace*{5\baselineskip}\Large\scshape\raggedright\bfseries}
{\hspace*{15px}+ }{0em}{}
[\normalsize\vspace*{-12px}\hspace*{15px}\_ \_ \_ \_ \_ \_ \_ \_ \_ \_ \_ \_ \hrulefill~\_ \_ \_ \_ \_\phantomsection]


\titlespacing{\subsubsection}{0px}{\baselineskip}{0px}
\titleformat{\subsubsection}
{\Needspace*{5\baselineskip}\Large\scshape\raggedright\bfseries}
{}{0em}{}
[\normalsize]





%%%%%%%%%%%%%%%%%%%%%%%%%%%%%%%%%%%%%%%%%%%%%%%%%%%%%%%%%%%%%%%%%%%%%%%%%%%%%%%
% document
%
%%%%%%%%%%%%%%%%%%%%%%%%%%%%%%%%%%%%%%%%%%%%%%%%%%%%%%%%%%%%%%%%%%%%%%%%%%%%%%
\begin{document}
	\hideFromPandoc
	\input{Epitech_template_modules-definition}

	%%%%%%%%%%%%%%%%%%%%%%%%%%%%%
	% background image
	%%%%%%%%%%%%%%%%%%%%%%%%%%%%%
	\bgimg

	%%%%%%%%%%%%%%%%%%%%%%%%%%%%%
	% presentation page
	%%%%%%%%%%%%%%%%%%%%%%%%%%%%%
	{\center
		\color{COLOR_main}
			\IfSubStr{\semester}{-1}
				{\textbf{\Huge{\moduleName}}}
				{\textbf{\Huge{B\semester{} - \moduleName}}}
				 \\	\vspace*{-0.2cm}
		\color{COLOR_light} 	\rule{0.7 \textwidth}{1pt}\\
		\color{COLOR_light} 	\LARGE{\moduleCode}\\
		\vspace*{6cm}
		\fontfamily{fco} \selectfont
		\color{COLOR_red}   	\textbf{\HUGE{\projectTitle}}\\	\vspace*{-0.5cm}
		\color{COLOR_redLight}	\rule{0.7 \textwidth}{1pt}\\
		\color{COLOR_redLight}	\LARGE{\projectSubtitle} \\
		\vspace*{2.5cm}
		\includegraphics[height=150px]{logo.png}\\
		\vfill \hfill
		\color{COLOR_light}
			 \small{1.1}
			
		\vspace*{0.2cm}
	}
	\newpage



	%%%%%%%%%%%%%%%%%%%%%%%%%%%%%
	% formalities
	%%%%%%%%%%%%%%%%%%%%%%%%%%%%%
	\addcontentsline{toc}{chapter}{\projectTitle}
	\phantomsection
	\addcontentsline{toc}{section}{\projectTitle}
	 
	\color{COLOR_dark}
	\hypertarget{description-du-programme}{%
\section{Description du programme}\label{description-du-programme}}

Le programme est composé de deux demi-journées de cours et de quatre
demi-journées de pratique de la programmation.

L'apprentissage s'effectue à travers l'utilisation des outils de
création et de modification du code ainsi qu'à travers la découverte
empirique de technologies de développement informatique.

\hypertarget{la-puxe9dagogie}{%
\subsection{La pédagogie}\label{la-puxe9dagogie}}

La pédagogie d'Epitech est axées autour de 3 compétences clés en
informatique : la \emph{rigueur}, l'\emph{autonomie} et le
\emph{professionnalisme}.

L'école inversée y est pratiquée : la connaissance n'a que très peu de
valeur ajoutée de nos jours, dû au fait qu'elle est très largement
disponible sur Internet, et qu'elle tombe très vite en désuétude dans le
domaine de l'informatique. A l'opposé, le travail de groupe, la
recherche personnelle, la capacité d'analyse sont les aptitudes
indispensables au traitement correct de cette quantité d'information
disponible, aptitudes qui permettront au professionnel de l'informatique
de se former continuellement sur les technologies en perpétuel
renouvellement. C'est pourquoi les activités sont déroulées sous la
forme de session de réflexion et de travail collectifs, au détriment des
cours traditionnels : les objectif sont de trouver, trier et utiliser
correctement la connaissance, et non de répéter celle (obsolète dans
quelques années tout au plus) de l'enseignant.

Enfin, nous ne pensons pas que la théorie soit un pré-requis à la
pratique, bien au contraire ! Le paradigme traditionnel consiste à
enseigner la théorie pour ensuite la mettre en pratique. Mais une
théorie décorrélée du réel n'engendre aucune motivation chez
l'apprenant, et semble souvent vide de sens\ldots{} Notre pédagogie se
fonde au contraire sur le principe que la théorie doit répondre à une
demande, motivée par la pratique. Le mode projet permet aux apprenants
de se confronter à des problématiques nouvelles, de chercher leurs
propres solutions en développant leur autonomie, leur capacité de
recherche et de synthèse, mais aussi leur créativité ; en cas d'échec,
la théorie trouve naturellement sa place en répondant aux questions
qu'ils ont été amenés à se poser par eux-mêmes.

Ce mode d'apprentissage, associé à une exigence de qualité et de rigueur
extrêmes, permet de former des professionnels compétents et adaptés aux
réalités du monde de l'informatique.

TABLE PLACEHOLDER!

\hypertarget{introduction}{%
\section{Introduction}\label{introduction}}

Initiation aux concepts fondamentaux (ligne de commande, programme, IDE,
langage, exécution).

Installation des outils nécessaires pour le cours (Node, npm, Git,
Heroku)

\hypertarget{jour-1}{%
\section{Jour 1}\label{jour-1}}

\hypertarget{matin}{%
\subsection{Matin}\label{matin}}

Introduction des concepts de base, présentation du programme

Découverte de différents concepts informatiques fondamentaux.

Recherche accompagnée sur les concepts, installation des différents
outils

\hypertarget{apruxe8s-midi}{%
\subsection{Après-midi}\label{apruxe8s-midi}}

Expérimentation interactive sur la base d'un programme fourni

Modification du code fourni et impact sur le programme

Outils pour compiler et tester du code

Outils pour partager du code. Principes de collaboration sur le code

\hypertarget{jour-2}{%
\section{Jour 2}\label{jour-2}}

\hypertarget{matin-1}{%
\subsection{Matin}\label{matin-1}}

Création d'un site web.

Utilisation de données issues d'une base de données locale.

Stockage des données et accès aux données créées par l'utilisateur.

\hypertarget{apruxe8s-midi-1}{%
\subsection{Après-midi}\label{apruxe8s-midi-1}}

Expérimentation sur le fonctionnement d'une application client/serveur

Déploiement d'un site internet en ligne.

Synthèse des différents concepts découverts

Retour et échange avec les apprenants sur la formation.
\end{document}
